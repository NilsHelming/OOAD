\documentclass[]{article}
\usepackage[utf8]{inputenc}
\usepackage[ngerman]{babel}
\usepackage[T1]{fontenc}
\usepackage{%
	ngerman,
	ae,
	times,  %% hier kann man die Schriftart einstellen
	graphicx,
	url,
	scrlayer-scrpage,
	lastpage,
	mathtools,
	geometry,
	multicol,
	cancel,
	xcolor,
	nicematrix,
	xfrac,
	tikz,
	pgfplots,
	amsmath,
	colortbl,
	centernot,
	dsfont,
	textgreek,
	icomma,
	pdfpages,
	kvmap,
	listings}
\usepackage[thinlines]{easybmat}
\usetikzlibrary{datavisualization}
\usetikzlibrary{datavisualization.formats.functions}
\usetikzlibrary{intersections}
\pgfplotsset{compat=1.17}
\newcommand{\del}[1]{\cancel{~#1~}}
\NiceMatrixOptions{ last-col,code-for-last-col = \color{blue}\scriptstyle,light-syntax}
\newlength\dlf
\newcommand\alignedhighlight[3]{
  % #1 = color
  % #2 = before alignment
  % #3 = after alignment
  &
  \begingroup
  \settowidth\dlf{$\displaystyle #2$}
  \addtolength\dlf{\fboxsep+\fboxrule}
  \hspace{-\dlf}
  \fcolorbox{#1}{#1}{$\displaystyle #2 #3$}
  \endgroup
}
\newcommand{\reference}[1]{ \text{\small{\textcolor{blue}{(#1)}}} }

\newcommand{\topic}{Objektorientierte Analyse und Design}
\newcommand{\subtopic}{Aufgabenblatt 01}
\newcommand{\authors}{Nils Helming, Lukas Lübberding}

%Head and Footnotes
\setlength{\headheight}{2.1\baselineskip} %baselineskip = minimum distance bbetween the bottom of one line to another.
\geometry{bottom = 3cm}
\setlength{\headsep}{\baselineskip}
\ihead[\topic\hrule]{\topic\hrule}
\chead[\subtopic\\~]{\subtopic\\~}
\ohead[\authors\\~]{\authors\\~}
\ifoot[~]{~}
\cfoot[~]{~}
\ofoot[Seite \thepage~von \pageref{LastPage}]{Seite \thepage~von \pageref{LastPage}}

%Paragraph spacings
\setlength{\parindent}{0em} %em = with of an 'M'
\setlength{\parskip}{1ex} %ex = height of an 'x'


\newcommand{\V}{\lor}
\newcommand{\A}{\land}
\newcommand{\T}[1]{\overline{#1}}
\newcommand{\eq}{\Leftrightarrow}
\newcommand{\rarr}{\Rightarrow}
\newcommand{\red}[1]{\textcolor{red}{#1}}

\newcommand{\unit}[1]{\text{#1}}
\newcommand{\fracunit}[2]{\frac{\unit{#1}}{\unit{#2}}}
\newcommand{\textsq}[1]{\ensuremath{\text{#1}^2}}
\newcommand{\textpow}[2]{\ensuremath{\text{#1}^{#2}}}
\newcommand{\tdot}{\ensuremath{\cdot}}


\begin{document}
% \includepdf[pages=-]{Modul01.pdf}
\section*{Aufgabe 1}
\subsection*{1 a)}
\lstinline|std::unique_ptr| ist ein Zeiger auf ein Objekt, wessen existenz allein auf diesen Zeiger beruht. Verliert dieser Zeiger den Verweis auf das Objekt, so wird das Objekt zerstört.

Wohingegen ein \lstinline|std::shared_ptr| mehrere Zeiger auf dasselbe Objekt erlaubt, sobald der letzte dieser Zeiger diesen Verweis verliert wird das Objekt zerstört.

\lstinline|std::weak_ptr| verhalten sich wie \lstinline|std::shared_ptr|, welche die Zerstörung des Objektes nicht verhindern, sobald also nur noch \lstinline|std::weak_ptr| auf das Objekt verweisen wird das Objekt gelöscht.

\lstinline|std::auto_ptr| ist eine veraltete Version des \lstinline|std::unique_ptr|. Aufgrund der Implementation dieser sind sie ausserdem nicht für die Verwendung in Containern wie Vektoren geeignet.

\lstinline|std::move| wird verwendet, um den \lstinline|std::unique_ptr| eines Objektes zu tauschen. Der alte Zeiger ist danach leer und der neue ist von dort an für die Speicherverwaltung dieses Objektes zuständig. Damit ist move für \lstinline|std::shared_ptr| nutzlos.

\lstinline|std::make_unique| und \lstinline|std::make_shared| erstellen sowohl die jeweligen Objekte, als auch die Zeiger darauf. Es gibt keine Funktion \lstinline|std::make_weak|, da ein so erstelltes Objekt direkt nach der Erstellung nur einen einzelnen \lstinline|std::weak_ptr| hat und damit sofort Zerstört werden sollte.

\subsection*{1 c)}
Der Verweis von Kunde zurück auf das Konto darf kein \lstinline|std::shared_ptr| sein, da dann immer mindestens ein \lstinline|std::shared_ptr| auf jedes Konto zeigt und die automatische Zerstörung von Konten durch \lstinline|std::shared_ptr| niemals in Kraft treten kann.

\end{document}